\newcommand{\anonsection}[1]{\section*{#1}\addcontentsline{toc}{section}{#1}}
\documentclass[10pt]{article}
\usepackage{a4wide}
\usepackage[utf8]{inputenc}
\usepackage[russian]{babel}
\usepackage{graphicx}
\usepackage{amsmath}
\usepackage{amsfonts}

\begin{document}

\thispagestyle{empty}

\begin{center}
\ \vspace{-3cm}

\includegraphics[width=0.5\textwidth]{msu.png}\\
{\scshape Московский государственный университет имени М.~В.~Ломоносова}\\
Факультет вычислительной математики и кибернетики\\
Кафедра системного анализа

\vfill

{\LARGE Отчет о практическом задании по курсу ``Оптимальное управление'' }

\vspace{1cm}

{\Huge\bfseries <<Решение линейной задачи быстродействия>>}
\end{center}

\vspace{1cm}

\begin{flushright}
  \large
  \textit{Студент 315 группы}\\
  Е.\,В.~Гуров

  \vspace{5mm}

  \textit{Руководитель практикума}\\
  к.ф.-м.н., доцент П.\,А.~Точилин
\end{flushright}

\vfill

\begin{center}
Москва, 2021
\end{center}
\newpage

\anonsection{Постановка задачи}
Задана линейная система обыкновенных дифференциальных уравнений:\\
\[ \dot{x} = Ax + Bu + f  \  , \ t \in [t_0, +\infty).\]
Здесь \(x \in \mathbb{R}^2, \ A \in \mathbb{R}^{2 \times 2}, \ B \in \mathbb{R}^{2 \times 2}, \ f \in \mathbb{R}^2, \ u \in \mathbb{R}^2 .\) На значения управляющих параметров \(u\) наложено ограничение: \(u \in \mathcal{P}.\) Пусть \( \mathcal{X}_0\) - начальное множество значений фазового вектора, \(\mathcal{X}_1\) - целевое множество значений фазового вектора. Необходимо решить задачу быстродействия, т.е. найти минимальное время \(T > 0\), за которое траектория системы, выпущенная в момент времени \(t_0\) из некоторой точки множества \(\mathcal{X}_0\), может попасть в некоторую точку множества \(\mathcal{X}_1\).
\bigskip
\[ \mathcal{P} = conv\{p_1, p_2, p_3, p_4\} \ , \ p_1, p_2, p_3, p_4 \in \mathbb{R}^2;\]
\[\mathcal{X}_0 = {x_0};\]
\[\mathcal{X}_1 = (\text{квадрат со стороной} \ r_1 > 0, \text{с центром в точке} \ x_1) + \] 
\[(\text{эллипсоид с матрицей конфигурации} \ Q = Q^T > 0  \ \text{и центром в нуле}).  \]

\anonsection{Теоретическая часть}
\textbf{Принцип максимума Понтрягина:}\bigskip \\
Основным утверждением, которым мы будем пользоваться при поиске оптимального управления, является принцип максимума Понтрягина для линейной задачи быстродействия. Он дает необходимые условия оптимальности для пары \( (u, x)\) - управления и соответствующей ему траектории системы.\bigskip \\
\textbf{Теорема.}(Принцип максимума Понтрягина)\smallskip\\
Пусть дана линейная задача быстродействия:
\[ \dot{x} = A(t)x + B(t)u + f(t) \ , \ t \in [t_0, t_1]\]
\[x \in \mathbb{R}^2 \ , \ u \in \mathcal{P} \subset \mathbb{R}^2 \]
\[ x(t_0) \in \mathcal{X}_0 \ , \ x(t_1) \in \mathcal{X}_1\]
\[t_1 - t_0 \to min\]
Пусть \( (x^*(t), u^*(t)) \) - оптимальная пара. Тогда существует \( \psi(t) \in \mathbb{R}^2\) - сопряженная переменная, являющаяся решением сопряженной задачи:
\[ \dot{\psi}(t) = -A^T(t) \psi(t) \ , \ \psi(t) \neq 0\]
и при этом выполнено:\medskip\\
1)Принцип максимума:
\( < \psi(t), B(t)u^*(t)>  \ = \rho(B(t) \mathcal{P}(t) , \psi(t))\)\medskip\\
2)Условие трансверсальности на левом конце:
\( < \psi(t_0), x^*(t_0)> \ = \rho(\mathcal{X}_0 , \psi(t_0))\)\medskip\\
3)Условие трансверсальности на правом конце:
\(<-\psi(t_1), x^*(t_1)> \ = \rho(\mathcal{X}_1 , -\psi(t_1)).\)
\newpage
\noindent\textbf{Вычисление опроных функций для множества начальных состояний, целевого множества и множества допустимых управлений:}\bigskip \\
Перечислим свойства, необходимые для вычисления требуемых опорных функций.\medskip\\
\textbf{Свойство} \(1^{\circ}\) (\textit{положительная однородность по первому аргументу}):
\[ \rho( \alpha X, \psi) = \alpha \cdot \rho(X, \psi) \ , \ \forall \alpha \geq 0 \ , \ \psi \in \mathbb{E}^n\]
\textbf{Свойство} \(2^{\circ}\) (\textit{аддитивность по первому аргументу}):
\[ \rho(X_1 + X_2, \psi) = \rho(X_1, \psi) + \rho(X_2, \psi)\]
\textbf{Свойство} \(3^{\circ}\) (\textit{опорная функция линейно преобразованного множества}):\\
пусть \(D\) - квадратная матрица \(n\) - го порядка, тогда
\[ \rho(DX, \psi) = \rho(X, D^T\psi) \ , \ \forall \psi \in \mathbb{E}^n\]
\textbf{Свойство} \(4^{\circ}\) (\textit{совпадение опорных функций множества и его наименьшей выпуклой оболочки}):
\[ \rho(X,\psi) = \rho(conv (X), \psi) \ , \ \forall \psi \in \mathbb{E}^n \bigskip \]
Перейдем теперь к подсчету функций множеств, фигурирующих в Принципе максимума.\medskip\\
\textbf{Множество} \( \mathcal{X}_0\):\smallskip\\
Опорная функция множества начальных состояний, состоящего из единственной точки \(x_0\) находится очевидно.
\[ \rho(\mathcal{X}_0, \psi) = \, <x_0, \psi> \ , \ \forall \psi \in \mathbb{E}^2\] 
\textbf{Множество} \( \mathcal{P}\):\smallskip\\
Пользуясь свойством 4, немедленно получаем опорную функцию множества допустимых управлений.
\[ \rho(\mathcal{P}, \psi) = \rho(conv(\{p_1, p_2, p_3, p_4\}), \psi) = \rho(\{p_1, p_2, p_3, p_4\}, \psi) = \max\limits_{i} <p_i, \psi> \ ,\ i = \overline{1, 4} \ , \ \forall \psi \in \mathbb{E}^2\] 
\textbf{Множество} \( \mathcal{X}_1\):\smallskip\\
Пользуясь свойствами 1, 2, 3 найдем опорную функцию целевого множества.\medskip\\
1)Опорная функция квадрата: 
\[ \rho (X_{r_1},\psi) = \rho(r_1 X_1, \psi) = r_1 \cdot \rho(X_1, \psi) = r_1 \cdot  \smash{\displaystyle\max_{x \in X_1}} ( x, \psi) = r_1 \cdot \smash{\displaystyle\max_{|x_1| \leq 1, |x_2| \leq 1}}(x_1 \psi_1 + x_2 \psi_2) =  r_1(|\psi_1| + |\psi_2|) \bigskip\]
2)Опорная функция эллипса с заданной матрицей конфигурации \(Q = Q^T > 0\) и центром в нуле:
\[ \rho(E_Q, \psi) = \rho(Q * B_1, \psi) = \rho(B_1, Q^T \psi) = \rho(B_1, Q\psi) = \|Q\psi\| \ , \ \forall \psi \in \mathbb{E}^2 \]
Здесь \( B_1 \) - это шар радиуса 1 с центром в точке 0. Опорная функция такого множества представляется в виде:
\[ \rho(B_1, \psi) = \max\limits_{x \in B} <x, \psi> \, = \, < \frac{\psi}{ \| \psi \| }, \psi> \, = \| \psi \| \ , \ \forall \psi \in \mathbb{E}^2\]
\newpage
\noindent3)Множество \( \mathcal{X}_1\) рпедставляет собой сумму трех множеств: точки \( x_1\), квадрата со стороной \(r_1\) и эллипса с матрицей конфигурации \( Q = Q^T > 0\). Его опорная функция представляется в виде:
\[ \rho(\mathcal{X}_1, \psi) = \rho(x_1, \psi) + \rho(X_{r_1}, \psi) + \rho(E_Q, \psi) = \, <x_1, \psi> + \, r_1 (|\psi_1| + |\psi_2|) + \|Q\psi\| \ , \ \forall \psi \in \mathbb{E}^2\]
\end{document}