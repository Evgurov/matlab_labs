\documentclass[11pt]{article}
\usepackage{a4wide}
\usepackage[utf8]{inputenc}
\usepackage[russian]{babel}
\usepackage{graphicx}
\usepackage{amsmath}

\begin{document}

\thispagestyle{empty}

\begin{center}
\ \vspace{-3cm}

\includegraphics[width=0.5\textwidth]{msu.png}\\
{\scshape Московский государственный университет имени М.~В.~Ломоносова}\\
Факультет вычислительной математики и кибернетики\\
Кафедра системного анализа

\vfill

{\LARGE Отчёт по практическому заданию}

\vspace{1cm}

{\Huge\bfseries <<Аппроксимация преобразования фурье с помощью БПФ в Matlab>>}
\end{center}

\vspace{1cm}

\begin{flushright}
  \large
  \textit{Студент 315 группы}\\
  Е.\,В.~Гуров

  \vspace{5mm}

  \textit{Руководитель практикума}\\
  к.ф.-м.н., доцент П.\,А.~Точилин
\end{flushright}

\vfill

\begin{center}
Москва, 2020
\end{center}
\newpage

\section{Постановка задачи}

С помощью Быстрого Преобразования Фурье (БПФ) получить апппроксимацию для преобразования фурье \(F(\lambda)\) следующих функций:~1) \( \frac{1 - cos^2(t)}{t} \), ~ 2) \( te^{-2t^2}\), ~ 3) \( \frac{2}{1 + 3t^6} \), ~ 4)\( e^{-5|t|}ln(3+t^4) \). Построить графики для действительной и мнимой частей получившейся аппроксимации, а так же рассчитать аналитически преобразования фурье для первых двух функций, и сравнить их с результатом работы БПФ(fft).

\section{Теоретическая часть}
\subsection{Изложение алгоритма работы программы}
Пусть $f(t)$ - исходная функция.\\
1) Ограничим область определения функции \(f(t)\) отрезком \([a,b]\)(рассмотрим произведение исходного сигнала на оконную функцию)
\[ h_{[a,b]}(t) =
  \begin{cases}
    1 & ,t \in [a,b]\\
    0 & ,t \notin [a,b]\\
  \end{cases}
\]
2)Далее производим пересчет шага дискретизации исходной функции:\\
\[ n = floor(T / step) + 1; \] 
\[ step = T / (n - 1); \]
Здесь \(T = T_{[a,b]}\) - окно входного сигнала соответствующее отрезку \([a,b]\) , step - шаг дискретизации, n - число отсчетов за время \(T\), \(floor\) - встроенная matlab-функция предназначенная для получения целой части числа.\\
3) Продлим функцию \(f\) по периоду \(T_{[a,b]}\). C помощью циклического сдвига массива значений исходной функции \(f\) получим сеточную функцию \(f(k \cdot  step) , k = 1, \dots, n\) на отрезке \([0, T_{[a,b]}]\). Этот массив подается на вход встроенной matlab-функции fft. На выходе имеем массив занчений апроксимации преобразования фурье для исходного сигнала. Построим его график с шагом дискретизации \( step_{F(\lambda)} = \frac{2 \pi}{T_[a,b]}\) и продлим по периоду \(T_{F(\lambda)} = n * step_{F(\lambda)}\), чтобы  изобразить на отрезке \([c,d]\) заданном в параметре выходного окна. 
\subsection{Получение аналитического выражения для преобразования фурье первых двух функций}
\bigskip
1. ~ \( f(t) =  \frac{1 - cos^2(t)}{t}\) \bigskip \\ 
\(f(t)\) - нечетная функция \(\implies Re \: F(\lambda) = 0\). Тогда с учетом \( e^{-i \lambda t} = cos(\lambda t) + i sin(\lambda t)\) получаем:\\
\( Im \: F(\lambda) = \int\limits_{-\infty}^{+\infty} \frac{1-cos^2(t)}{t} (-i sin(\lambda t)) \, \mathrm{d}t = -i \int\limits_{-\infty}^{+\infty} \frac{1 - cos^2(t)}{t} sin(\lambda t) \, \mathrm{d}t = -i \int\limits_{-\infty}^{+\infty} \frac{sin^2(t)}{t} sin(\lambda t) \, \mathrm{d}t =\\
= -i \int\limits_{-\infty}^{+\infty} sin(t) \frac{sin(t) \cdot sin(\lambda t)}{t} \, \mathrm{d}t = -i \int\limits_{-\infty}^{+\infty} sin(t) \frac{cos(t - \lambda t) - cos(t + \lambda t)}{2t} \, \mathrm{d}t =\\
= -\frac{i}{2} \int\limits_{-\infty}^{+\infty}  \frac{sin(t) cos[(1-\lambda)t] - sin(t) cos[(1+\lambda)t]}{t} \, \mathrm{d}t = -\frac{i}{2} \int\limits_{-\infty}^{+\infty}\frac{sin[(2-\lambda)t] + sin(\lambda t) - sin[(2+\lambda) t] - sin(-\lambda t)}{2t} \, \mathrm{d}t  =\\
= -\frac{i}{4} \int\limits_{-\infty}^{+\infty} \frac{sin[(2-\lambda)t] + 2sin(\lambda t) - sin[(2+\lambda)t]}{t} \, \mathrm{d}t = -\frac{i}{2} \int\limits_{0}^{+\infty} \frac{sin[(2-\lambda)t] + 2sin(\lambda t) - sin[(2+\lambda)t]}{t} \, \mathrm{d}t =\\ 
\bigskip 
= \Big\{\int\limits_{0}^{+\infty} \frac{sin(at)}{t} \, \mathrm{d}t = \frac{\pi}{2} sgn(a) \) - интеграл Дирихле\(\Big\} = \frac{i \pi}{4} \Big[ sgn(\lambda - 2) - 2sgn(\lambda) + sgn(\lambda + 2)\Big] \) \\
\bigskip
2. ~ \(f(t) = te^{-2t^2}\) \\
\(f(t)\) - нечетная функция \(\implies Re \: F(\lambda) = 0\). Аналогично имеем:\\
\(Im \: F(\lambda) = \int\limits_{-\infty}^{+\infty} te^{-2t^2} \big(-i sin(\lambda t)\big) \, \mathrm{d}t = -i \int\limits_{-\infty}^{+\infty} te^{-2t^2} sin(\lambda t) \, \mathrm{d}t = \frac{i}{4} \int\limits_{-\infty}^{+\infty} sin(\lambda t) \, \mathrm{d}(e^{-2t^2}) =\\
= \frac{i}{4}\big[ sin(\lambda t) e^{-2t^2}\big|_{-\infty}^{+\infty} - \lambda \int\limits_{-\infty}^{+\infty} cos(\lambda t) e^{-2t^2} \, \mathrm{d}t \big] = -\frac{i \lambda}{4} \int\limits_{-\infty}^{+\infty} cos(\lambda t) e^{-2t^2} \, \mathrm{d}t\) \bigskip \\
Найдем значение последнего интеграла как функции от \(\lambda: \bigskip \\
\int\limits_{-\infty}^{+\infty}cos(\lambda t) e^{-2t^2} \, \mathrm{d}t = \int\limits_{-\infty}^{+\infty} Re \: (e^{-2t^2} \cdot e^{i \lambda t}) \, \mathrm{d}t = Re \int\limits_{-\infty}^{+\infty} e^{-2t^2} \cdot e^{i \lambda t} \, \mathrm{d}t = Re \int\limits_{-\infty}^{+\infty} e^{-2t^2 + i \lambda t} \, \mathrm{d}t = \\ 
= Re \int\limits_{-\infty}^{+\infty} e^{-(\sqrt{2} t - \frac{i \lambda}{2\sqrt{2}})^2 - \frac{\lambda^2}{8}} \, \mathrm{d}t = Re \: \Big( e^{-\frac{\lambda^2}{8}} \int\limits_{-\infty}^{+\infty} e^{-(\sqrt{2} t - \frac{i \lambda}{2\sqrt{2}})^2} \mathrm{d}t \Big) = \\
= Re \: \Big[ e^{-\frac{\lambda^2}{8}} \cdot \frac{1}{\sqrt{2}} \int\limits_{-\infty}^{+\infty} e^{-(\sqrt{2} t - \frac{i \lambda}{2\sqrt{2}})^2} \mathrm{d}(\sqrt{2} t - \frac{i \lambda}{2\sqrt{2}})\Big] = \Big\{ \int\limits_{-\infty}^{+\infty} e^{-t^2} \mathrm{d}t = \sqrt{\pi} \) - интеграл Пуассона\(\Big\} = \sqrt{\frac{\pi}{2}} e^{-\frac{\lambda^2}{8}}\) \bigskip \\
В итоге получим: \(F(\lambda) = -\frac{i \lambda}{4} \sqrt{\frac{\pi}{2}} e^{-\frac{\lambda^2}{8}}\)

\section{Графики преобразований фурье данных функций при разных значениях параметров}
Напомним данные функции:~1) \( \frac{1 - cos^2(t)}{t} \), ~ 2) \( te^{-2t^2}\), ~ 3) \( \frac{2}{1 + 3t^6} \), ~ 4) \( e^{-5|t|}ln(3+t^4) \). \bigskip \\
Далее на графиках синим цветом изображен результат работы функции fft, красным - аналитически посчитанное преобразование фурье, а зеленым - сумма аналитеских \(F(\lambda)\), изображенных с соответствующими сдвигами. \bigskip \\
\newpage
\begin{itemize}
	\item \textbf{Эффект наложения спектра} \\
	\begin{itemize}
		\item \textbf{Функция №2.} Входное окно: \textbf{[-100, 100]}, Шаг дискретизации: \textbf{0.1} \medskip \\
			Видно, что сумма аналитических значений преобразований фурье, изображенных с определенными сдвигами отличается от реального преобразования фурье в местах наложения.\bigskip \\
			\includegraphics[width=0.8\textwidth]{aliasing_re.png}\\
			\includegraphics[width=0.8\textwidth]{aliasing.png}\\
		\item \textbf{Функция №2.} Входное окно: \textbf{[-100, 100]}, Шаг дискретизации: \textbf{0.06} \medskip \\
			При уменьшении шага дискретизации эффект пропадает практически полностью. \bigskip \\
			\includegraphics[width=0.8\textwidth]{no_aliasing_re.png}\\
			\includegraphics[width=0.8\textwidth]{no_aliasing.png}\\
	\end{itemize} 
	\newpage
	\item \textbf{Эффект ряби} \\
	\begin{itemize}
		\item \textbf{Функция №2.} Входное окно: \textbf{[-0.7, 20]}, Шаг дискретизации: \textbf{0.05} \medskip \\
			В случае если окно обрезает функцию в точках, где она значительно отличается от нуля, на вход fft подается ``сильно'' разрывная функция. Из за чего возникает эффект Гиббса(ряби).\bigskip \\
			\includegraphics[width=0.8\textwidth]{ripple_effect_0.05_re}\\
			\includegraphics[width=0.8\textwidth]{ripple_effect_0.05_im}\\
		\item \textbf{Функция №2.} Входное окно: \textbf{[-0.7, 20]}, Шаг дискретизации: \textbf{0.001} \medskip \\
			При уменьшении шага дискретизации эффект ряби не исчезает. С помощью изменения частоты измерений ээфект ряби устранить принципиально нельзя. \bigskip \\
			\includegraphics[width=0.8\textwidth]{ripple_effect_0.001_re}\\
			\includegraphics[width=0.8\textwidth]{ripple_effect_0.001_im}\\
	\end{itemize} 
	\newpage
	\item \textbf{Графики каждого преобразования Фурье при разных параметрах}
	\begin{itemize}
		\item \textbf{Функция №1}
			\begin{itemize}	
				\item Входное окно: \textbf{[1, 100]}, Шаг дискретизации: \textbf{0.1} \medskip \\
					\includegraphics[width=0.7\textwidth]{func1_1_100_0.1_re.png}\\
					\includegraphics[width=0.7\textwidth]{func1_1_100_0.1_im.png}\\
				\item Входное окно: \textbf{[0.001, 100]}, Шаг дискретизации: \textbf{0.01} \medskip \\
					\includegraphics[width=0.7\textwidth]{func1_0.001_100_0.01_re.png}\\
					\includegraphics[width=0.7\textwidth]{func1_0.001_100_0.01_im.png}\\
			\end{itemize}
		\item \textbf{Функция №2}	
			\begin{itemize}
				\item Входное окно: \textbf{[-100, 100]}, Шаг дискретизации: \textbf{0.1} \medskip \\
					\includegraphics[width=0.7\textwidth]{func2_-100_100_0.1_re.png}\\
					\includegraphics[width=0.7\textwidth]{func2_-100_100_0.1_im.png}\\
				\item Входное окно: \textbf{[0.001, 100]}, Шаг дискретизации: \textbf{0.01} \medskip \\
					\includegraphics[width=0.7\textwidth]{func2_-20_20_0.1_re.png}\\
					\includegraphics[width=0.7\textwidth]{func2_-20_20_0.1_im.png}\\
			\end{itemize}
		\item \textbf{Функция №3}
			\begin{itemize}	
				\item Входное окно: \textbf{[-10, 10]}, Шаг дискретизации: \textbf{0.1} \medskip \\
					\includegraphics[width=0.7\textwidth]{func3_-10_10_0.1_re.png}\\
					\includegraphics[width=0.7\textwidth]{func3_-10_10_0.1_im.png}\\
				\item Входное окно: \textbf{[-10, 10]}, Шаг дискретизации: \textbf{0.05} \medskip \\
					\includegraphics[width=0.7\textwidth]{func3_-10_10_0.05_re.png}\\
					\includegraphics[width=0.7\textwidth]{func3_-10_10_0.05_im.png}\\
					\newpage
				\item Входное окно: \textbf{[-10, 10]}, Шаг дискретизации: \textbf{0.01} \medskip \\
					\includegraphics[width=0.7\textwidth]{func3_-10_10_0.01_re.png}\\
					\includegraphics[width=0.7\textwidth]{func3_-10_10_0.01_im.png}\\
			\end{itemize}
		\item \textbf{Функция №4}	
			\begin{itemize}
				\item Входное окно: \textbf{[-100, 100]}, Шаг дискретизации: \textbf{0.5} \medskip \\
					\includegraphics[width=0.7\textwidth]{func4_-100_100_0.5_re.png}\\
					\includegraphics[width=0.7\textwidth]{func4_-100_100_0.5_im.png}\\
				\item Входное окно: \textbf{[-10, 10]}, Шаг дискретизации: \textbf{0.1} \medskip \\
					\includegraphics[width=0.7\textwidth]{func4_-10_10_0.1_re.png}\\
					\includegraphics[width=0.7\textwidth]{func4_-10_10_0.1_im.png}\\
			\end{itemize}
	\end{itemize}
\end{itemize}

\end{document}

